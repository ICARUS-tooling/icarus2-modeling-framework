%
% File icqp_specification.tex
%

\documentclass[11pt,a4paper]{report}

\usepackage{icarus2}

\title{ICARUS2 Corpus Query Processor Specification}

\author{Markus Gärtner}

\date{2020}

\begin{document}

\maketitle

\tableofcontents

\listoffigures

\listoftables

\listofalgorithms 

\newpage

\chapter*{Introduction}
\label{sec:intro}

The \ac{icqp} is a custom evaluation engine for corpus queries that follow the \ac{iql} specification.

\chapter{Notations and Definitions}
\label{chap:definitions}

\begin{description}
	\item[sequence] xxx
	\item[tree] xxx
	\item[search node] a \textit{search node} $n$ is a tuple $(c,\delta_p,\delta_s)$ where $c$ is the local constraint predicate taking as argument a target node $n_i$ from a sequence, $\delta_p$ is the minimum distance to the node's predecessor and  
	\item[tree search node] a \textit{tree search node} $v$ is a tuple $(c,l,)$
	\item[sequence] xxx
	\item[sequence] xxx
	\item[sequence] xxx
	\item[sequence] xxx
	\item[sequence] xxx
\end{description}


\section{Tree Inclusion}
\label{sec:tree-inclusion}

Let $T$ be a \textit{rooted} tree. 
We say that $T$ is \textit{annotated} if each node $v$ in $T$ is assigned a set of annotations $a_1(v)..a_n(v)$ where each annotation function $a_i$ provides its own alphabet $\Sigma_i$ of available annotation values.
This definition is similar to the basic notion of a \textit{labeled} tree where each node is assigned a single label or value from a common alphabet $\Sigma$, but to accommodate the nature of multi-layer annotations in linguistic corpora, we extend this definition.
$T$ is further \textit{ordered} if for any node $v$ its children $v_1..v_n$ follow a globally consistent ordering scheme.
If not specified otherwise, all trees in this document are \textit{rooted} and \textit{annotated}.

A tree $P$ is said to be \textit{included} in $T$, denoted $P \sqsubseteq T$, if deleting nodes in $T$ can yield $P$.
Deleting a node $v$ in $T$ means replacing $v$ with the sequence of its children.
Solving the tree inclusion problem means determining if $P$ actually is included in $T$ and then also returning all (or up to a specified number of) subtrees of $T$ that include $P$.

\chapter{Plain Matching}
\label{chap:plain-matching}

\chapter{Sequence Matching}
\label{chap:sequence-matching}

Elements participating in sequence matching:
\begin{description}
	\item[node] singular and optionally quantified element
	\item[sequence] sequence of elements that adhere to given arrangement
	\item[grouping] sequence of elements that is optionally quantified as a whole
	\item[disjunction] two or more alternative elements
\end{description}

\section{Definitions}
\label{sec:seq-def}

Let $L$ be a list, then $N_L = |L|$ is its length and $l_{i} \in L$ denotes the element at position $i$ of the list where $1 \leq  i \leq N_L$.
Let $T$ be the list of target elements and $S$ the tree of search nodes.

Utility procedures and functions used in the algorithms of this section:
\begin{description}
	\item[atom($s$)] Returns the single wrapped child node for $s$. 
	\item[child($s, i$)] Returns the child node $s_i$ on index $i$ for $s$. 
	\item[eval($s,t$)] Evaluates the inner constraints of search node $s$ on the target item $t$. The result is a cache entry with a boolean value indicating whether the evaluation was successful.
	\item[cacheGet($s,t$)] Retrieve a cached entry for the evaluation of search node $s$ on target element $t$. If no entry exist, $nil$ is returned.
	\item[cacheSet($s, t, entry$)] Stores the given $entry$ in the cache for the evaluation of search node $s$ on target element $t$.  
	\item[next($s$)] Returns the next search node in the sequence after $s$.
	\item[size($s$)] Returns the number of child nodes attached to $s$.
	\item[type($s$)] Returns the type of the given search node, one of the following:\footnote{Nested nodes can be obtained with $atom(s)$ for single embeddings or $atom(s,i)$ for indexed elements where $atoms(s)$ returns the number of embedded nodes and $1 \leq i \leq atom(s)$.}
	\begin{description}
		\item[$single$] A single atomic node that is existentially quantified.
		\item[$sequence$] A group of adjacent search nodes.
		\item[$group$] A group of ordered (but not necessarily adjacent) search nodes. 
		\item[$repetition$] A quantifier $(c_{min},c_{max})$ and an associated $atom$ search node.
		\item[$negation$] An embedded node that must not match.
		\item[$universal$] An embedded node that is expected to match all available target elements. Note that universal quantification is only allowed if the node is the only one in the (resolved) global context.
		\item[$choice$] A group of nodes that represent logical alternatives.
		\item[$region$] An embedded node and associated index interval $i_{start} = first(s)$ to $i_{end}=last(s)$ of legal positions for matching.
		\item[$spot$] An embedded node and a single associated index for matching.
	\end{description}
\item[VALUE($entry$)] Extracts the actual boolean result value from a cache entry.
\end{description}

\section{Rules}
\label{sec:seq-rules}

\tikzset{
	search node/.style={rectangle, draw=black!50, thick}
}

\newcommand{\rNode}[4]{\node[rectangle, draw=black!50, thick](#3) at(#1,#2) {#4}}

This section describes the recursive rules for constructing the object graph of search nodes from the original elements in the query.
All rules assume the existence of already processed nodes that resulted in some \textit{head} part to the left that is connected to a \textit{tail} to the left, as illustrated below:
\begin{center}
	\begin{tikzpicture}\centering
		\rNode{1}{0}{h}{.. head};
		\rNode{4}{0}{t}{tail ..};
		\draw[->] (h) -- (t);
	\end{tikzpicture}
\end{center}
Initially the \textit{head} node is just a generic entry point and the \textit{tail} simply implements the final acceptance node that ensures that all actual search nodes in the query have been evaluated already.

The elements in the query are effectively processed left to right and for every encountered element a new node is inserted between the current \textit{head} and \textit{tail}: %TODO reformulate, LtR is oversimplifying it
\begin{center}
	\begin{tikzpicture}\centering
		\node[search node](h) at(0,0) {.. head};
		\node[search node](t) at(4,0) {tail ..};
		\draw[->] (h) -- (t);
		\node[rectangle, fit=(h)(t)](group){};
		\node[draw](element) [below=of group] {element};
		\draw[->] (element) -- (group);
	\end{tikzpicture}
\end{center}
This process is recursive and builds the graph of interconnected search nodes used in the algorithms described in the respective sub sections.

\subsection{Single Node}
\label{sec:seq-single}


\begin{algorithm}[H]
	\caption[Constraint matching]{Local constraint matching with memoization to prevent redundant calls to costly constraint evaluation.}
	\label{alg:local-constraint}
	\begin{algorithmic}[1]
		\Procedure{LOCAL-MATCH}{$s, t$}
			\State $entry \gets $ \Call{cacheGet}{$s,t$}
			\If{$entry = nil$}
				\State $entry \gets$ \Call{eval}{$s,t$}
				\State \Call{cacheSet}{$s, t, entry}$
			\EndIf
			\State \textbf{return} \Call{value}{$entry$}
		\EndProcedure
	\end{algorithmic}
\end{algorithm}

\begin{algorithm}[H]
	\caption{Single node matching}
	\label{alg:match-node}
	\begin{algorithmic}[1]
		\Procedure{MATCH-NODE}{$s, T, j_{start}, j_{max}$} \Comment{$j_{start} \leq j_{max}$}
			\State $j \gets j_{start}$
			\While{$j <= j_{max}$}
				\If{\Call{LOCAL-MATCH}{$s,t_j$}}
					\State $tails \gets next(s)$
					\State \textbf{return} \Call{MATCH}{$tail, T, j, j_{max}$}
				\EndIf
				\State $j \gets j + 1$
			\EndWhile
			\State \textbf{return} $false$
		\EndProcedure
	\end{algorithmic}
\end{algorithm}

\section{Algorithm}
\label{sec:seq-alg}

\begin{algorithm}[H]
	\caption{Generic node matching}
	\label{alg:match}
	\begin{algorithmic}[1]
		\Procedure{MATCH}{$s, T, j_{start}, j_{max}$} \Comment{$j_{start} \leq j_{max}$}
			\State $type_s \gets$ \Call{type}{$s$}
			\If{$type_s = single$}
			\State \textbf{return} $j$
			\EndIf
			\State $j \gets j + 1$
			\State \textbf{return} $false$
		\EndProcedure
	\end{algorithmic}
\end{algorithm}

\begin{algorithm}[H]
	\caption{Adjacent sequence matching}
	\label{alg:match-adjacent}
	\begin{algorithmic}[1]
		\Procedure{MATCH-ADJACENT}{$C, T, j_{start}, j_{max}$} \Comment{$j_{start} \leq j_{max}, C \subseteq S$}
		\State $j \gets j_{start}$
		\While{$j <= j_{max}$}
		\If{$\mathrm{LOCAL\_MATCH}(s,t_j)$}
		\State \textbf{return} $j$
		\EndIf
		\State $j \gets j + 1$
		\EndWhile
		\State \textbf{return} $-1$
		\EndProcedure
	\end{algorithmic}
\end{algorithm}

\begin{algorithm}[H]
	\caption{Quantified matching}
	\label{alg:match-quantified}
	\begin{algorithmic}[1]
		\Procedure{MATCH-QUANTIFIED}{$s, T, j_{start}, j_{max}$} \Comment{$j_{start} \leq j_{max}$}
		\State $j \gets j_{start}$
		\While{$j <= j_{max}$}
		\If{$\mathrm{LOCAL\_MATCH}(s,t_j)$}
		\State \textbf{return} $j$
		\EndIf
		\State $j \gets j + 1$
		\EndWhile
		\State \textbf{return} $-1$
		\EndProcedure
	\end{algorithmic}
\end{algorithm}


%TODO intervals per node

\chapter{Tree Matching}
\label{chap:tree-matching}

\begin{appendices}

\end{appendices}
\end{document}